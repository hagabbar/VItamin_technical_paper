%% Template for a preprint Letter or Article for submission
%% to the journal Nature.
%%

\documentclass[%
%superscriptaddress,
%groupedaddress,
%unsortedaddress,
%runinaddress,
%frontmatterverbose, 
%preprint,
showpacs,
%preprintnumbers,
%nofootinbib,
%nobibnotes,
%bibnotes,
 amsmath,amssymb,
 aps,
 twocolumn,
 prl,
 reprint,
%pra,
%prb,
%rmp,
%prstab,
%prstper,
floatfix,
]{revtex4-1}

\usepackage{graphicx}% Include figure files
\usepackage{dcolumn}% Align table columns on decimal point
\usepackage{bm}% bold math
%\usepackage{lineno}
\usepackage{amsmath}
\usepackage{amsfonts}
\usepackage{amssymb}
\usepackage{color}
\usepackage{acronym}
\usepackage{multirow}
\usepackage{tabularx}
\usepackage{hyperref}
%\usepackage{booktabs}

\hypersetup{
%--- fill inside borders ---
  colorlinks=true,        % false: boxed links; true: colored links
  linkcolor=black,         % color of internal links
  citecolor=cyan,         % color of links to bibliography
}

%% ----- comment commands for each of us
\newcommand{\chris}[1]{\textbf{\textcolor{red}{CHRIS: #1}}}
\newcommand{\francesco}[1]{\textbf{\textcolor{green}{FRANCESCO: #1}}}
\newcommand{\hunter}[1]{\textbf{\textcolor{blue}{HUNTER: #1}}}
\newcommand{\siong}[1]{\textbf{\textcolor{cyan}{SIONG: #1}}}
\newcommand{\rod}[1]{\textbf{\textcolor{yellow}{ROD: #1}}}

\begin{document}

\preprint{APS/123-QED}

\title{VItamin: an in-depth analysis of the tool used for gravitational wave Bayesian parameter estimation}

\author{Hunter Gabbard$^1$}
 \email{Corresponding author: h.gabbard.1@research.gla.ac.uk}
\author{Chris Messenger$^1$}
\author{Ik Siong Heng$^1$}
\author{Francesco Tonolini$^2$}
\author{Roderick Murray-Smith$^2$}

\affiliation{
 SUPA, School of Physics and Astronomy$^1$, \\
 University of Glasgow, \\
 Glasgow G12 8QQ, United Kingdom \\ \\
 School of Computing Science$^2$, \\
 University of Glasgow, \\
 Glasgow G12 8QQ, United Kingdom \\
}

\date{\today}

\begin{abstract}
abstract
\end{abstract}

\maketitle

% TO-DO LIST
%

\acrodef{GW}[GW]{Gravitational wave}
\acrodef{BBH}[BBH]{binary black hole}
\acrodef{EM}[EM]{electromagnetic}
\acrodef{CBC}[CBC]{compact binary coalescence}
\acrodef{BNS}[BNS]{binary neutron star}
\acrodef{NSBH}[NSBH]{neutron star black hole}
\acrodef{PSD}[PSD]{power spectral density}
\acrodef{ELBO}[ELBO]{evidence lower bound}
\acrodef{LIGO}[LIGO]{advanced Laser Interferometer Gravitational wave Observatory}
\acrodef{CVAE}[CVAE]{conditional variational autoencoder}
\acrodef{KL}[KL]{Kullback--§Leibler}
\acrodef{GPU}[GPU]{graphics processing unit}
\acrodef{LVC}[LVC]{LIGO-Virgo Collaboration}
\acrodef{PP}[p-p]{probability-probability}

%%%%%%%%%%%%%%%%%%%%%%%%%%%%%%%%%%%%%%%%%%%%%%%%%%%%%%%%%%%%
%%%%%%%%%%%%%%%%%%%%%%%%%%%%%%%%%%%%%%%%%%%%%%%%%%%%%%%%%%%%
%%%%%%%%%%%%%%%%%%%%%%%%%%%%%%%%%%%%%%%%%%%%%%%%%%%%%%%%%%%%
\section{Introduction}

\begin{itemize}
\item The state of the GW field and the ongoing detections
\item GW PE is a key aspect of GW astrophysics
\item ML is rapidly becoming more and more relevant
\item Here is the what this paper is all about
\item Here is the paper structure
\end{itemize}

%%%%%%%%%%%%%%%%%%%%%%%%%%%%%%%%%%%%%%%%%%%%%%%%%%%%%%%%%%%%
%%%%%%%%%%%%%%%%%%%%%%%%%%%%%%%%%%%%%%%%%%%%%%%%%%%%%%%%%%%%
%%%%%%%%%%%%%%%%%%%%%%%%%%%%%%%%%%%%%%%%%%%%%%%%%%%%%%%%%%%%
\section{VItamin}

%%%%%%%%%%%%%%%%%%%%%%%%%%%%%%%%%%%%%%%%%%%%%%%%%%%%%%%%%%%%
%%%%%%%%%%%%%%%%%%%%%%%%%%%%%%%%%%%%%%%%%%%%%%%%%%%%%%%%%%%%
\subsection{What is a conditional variational autoencoder?}

%%%%%%%%%%%%%%%%%%%%%%%%%%%%%%%%%%%%%%%%%%%%%%%%%%%%%%%%%%%%
%%%%%%%%%%%%%%%%%%%%%%%%%%%%%%%%%%%%%%%%%%%%%%%%%%%%%%%%%%%%
\subsection{The specific implementation for VItamin}

\begin{itemize}
\item Descriptions and systematic studies related to each component of the
network.
\item What is the latent space? What does it look like?
\item Does the ELBO have a meaning?
\end{itemize}

%%%%%%%%%%%%%%%%%%%%%%%%%%%%%%%%%%%%%%%%%%%%%%%%%%%%%%%%%%%%
%%%%%%%%%%%%%%%%%%%%%%%%%%%%%%%%%%%%%%%%%%%%%%%%%%%%%%%%%%%%
\subsection{The training process}

%%%%%%%%%%%%%%%%%%%%%%%%%%%%%%%%%%%%%%%%%%%%%%%%%%%%%%%%%%%%
%%%%%%%%%%%%%%%%%%%%%%%%%%%%%%%%%%%%%%%%%%%%%%%%%%%%%%%%%%%%
\subsection{The testing process - application to data}

\begin{itemize}
\item lots of benchmarking for speed and memory usage
\end{itemize}

%%%%%%%%%%%%%%%%%%%%%%%%%%%%%%%%%%%%%%%%%%%%%%%%%%%%%%%%%%%%
%%%%%%%%%%%%%%%%%%%%%%%%%%%%%%%%%%%%%%%%%%%%%%%%%%%%%%%%%%%%
%%%%%%%%%%%%%%%%%%%%%%%%%%%%%%%%%%%%%%%%%%%%%%%%%%%%%%%%%%%%
\section{The baseline configuration}

\begin{itemize}
\item Use this time to define a baseline version of VItamin that will serve as
the benchmark for all subsequent tests
\item I advise a BBH case at 2kHz sampling using 4 sec segments and the full 15
dimensional parameter space for 3 detectors.
\item Show some example results
\end{itemize}

%%%%%%%%%%%%%%%%%%%%%%%%%%%%%%%%%%%%%%%%%%%%%%%%%%%%%%%%%%%%
%%%%%%%%%%%%%%%%%%%%%%%%%%%%%%%%%%%%%%%%%%%%%%%%%%%%%%%%%%%%
%%%%%%%%%%%%%%%%%%%%%%%%%%%%%%%%%%%%%%%%%%%%%%%%%%%%%%%%%%%%
\section{New developments}

%%%%%%%%%%%%%%%%%%%%%%%%%%%%%%%%%%%%%%%%%%%%%%%%%%%%%%%%%%%%
%%%%%%%%%%%%%%%%%%%%%%%%%%%%%%%%%%%%%%%%%%%%%%%%%%%%%%%%%%%%
\subsection{Importance sampling}

%%%%%%%%%%%%%%%%%%%%%%%%%%%%%%%%%%%%%%%%%%%%%%%%%%%%%%%%%%%%
%%%%%%%%%%%%%%%%%%%%%%%%%%%%%%%%%%%%%%%%%%%%%%%%%%%%%%%%%%%%
\subsection{Application of CVAE in different scenarios}

%%%%%%%%%%%%%%%%%%%%%%%%%%%%%%%%%%%%%%%%%%%%%%%%%%%%%%%%%%%%
%%%%%%%%%%%%%%%%%%%%%%%%%%%%%%%%%%%%%%%%%%%%%%%%%%%%%%%%%%%%
\subsection{Gaussian vs. Non-Gaussian noise}

%%%%%%%%%%%%%%%%%%%%%%%%%%%%%%%%%%%%%%%%%%%%%%%%%%%%%%%%%%%%
%%%%%%%%%%%%%%%%%%%%%%%%%%%%%%%%%%%%%%%%%%%%%%%%%%%%%%%%%%%%
\subsection{Waveform systematics}

%%%%%%%%%%%%%%%%%%%%%%%%%%%%%%%%%%%%%%%%%%%%%%%%%%%%%%%%%%%%
%%%%%%%%%%%%%%%%%%%%%%%%%%%%%%%%%%%%%%%%%%%%%%%%%%%%%%%%%%%%
\subsection{Transfer learning - changing PSD}

%%%%%%%%%%%%%%%%%%%%%%%%%%%%%%%%%%%%%%%%%%%%%%%%%%%%%%%%%%%%
%%%%%%%%%%%%%%%%%%%%%%%%%%%%%%%%%%%%%%%%%%%%%%%%%%%%%%%%%%%%
\subsection{conditional on the PSD}

\begin{itemize}
\item We can add PSD parameters to the $x$-vector and infer them, or...
\item We can add PSD parameters to the $y$-vector and feed them in at testing
time so you get an answer conditional on the PSD you choose.
\end{itemize}

%%%%%%%%%%%%%%%%%%%%%%%%%%%%%%%%%%%%%%%%%%%%%%%%%%%%%%%%%%%%
%%%%%%%%%%%%%%%%%%%%%%%%%%%%%%%%%%%%%%%%%%%%%%%%%%%%%%%%%%%%
%%%%%%%%%%%%%%%%%%%%%%%%%%%%%%%%%%%%%%%%%%%%%%%%%%%%%%%%%%%%
\section{O2 events}

\begin{itemize}
\item This section could be a separate paper.
\item We provide results for all O2 BBH detections.
\end{itemize}

%%%%%%%%%%%%%%%%%%%%%%%%%%%%%%%%%%%%%%%%%%%%%%%%%%%%%%%%%%%%
%%%%%%%%%%%%%%%%%%%%%%%%%%%%%%%%%%%%%%%%%%%%%%%%%%%%%%%%%%%%
%%%%%%%%%%%%%%%%%%%%%%%%%%%%%%%%%%%%%%%%%%%%%%%%%%%%%%%%%%%%
\section{Conclusions}

\begin{itemize}
\item Summarise the paper
\item Highlight the best parts
\item Highlight the worst parts
\item End on a positive note.
\end{itemize}

\bibliographystyle{apsrev4-1}
\bibliography{references}% Produces the bibliography via BibTeX.

\end{document}
